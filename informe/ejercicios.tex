\section{Ejercicio 1}

\emph{Completar la función broadcast\_block para que comunique a todos los demás nodos el nuevo bloque creado con el tag \texttt{TAG\_NEW\_BLOCK}. Cada nodo debe enviar los mensajes en un orden distinto a los demás nodos.}

Para que se envíen los mensajes en un orden distinto cada nodo hace el envío respetando el orden de los rangos, comenzando desde el nodo inmediatamente siguiente y ``dando la vuelta"\ para seguir con el de menor rango una vez que se llega al nodo de máximo rango. Es decir, el nodo de rango $i$ envía los mensajes en el orden de rangos $i+1 \mod N,\ i+2 \mod N,\ \dots,\ i+(N-1) \mod N$, donde $N$ es la cantidad de nodos.

Los mensajes se envían de forma no bloqueante con \texttt{MPI\_Isend} para que la lectura del bloque se pueda hacer concurrentemente. Si se usara \texttt{MPI\_Send}, cada pedido de envío comenzaría recién una vez que el llamado anterior haya terminado de leer el bloque, forzando una lectura secuencial.

Luego de hacer todos los pedidos de envío se espera a que los mismos finalicen con un llamado a \texttt{MPI\_Waitall}.